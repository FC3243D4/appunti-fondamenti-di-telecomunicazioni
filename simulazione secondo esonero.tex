\documentclass{article}
\usepackage{amsfonts}
\usepackage{amsmath}
\usepackage{amssymb}
\usepackage{mathrsfs}
\usepackage{extarrows}
\usepackage{hyperref}
\usepackage[utf8]{inputenc}
\usepackage{graphicx}
\usepackage{mathtools}
\usepackage[a4paper, total={6in, 8in}]{geometry}
\usepackage[table]{xcolor}
\usepackage{tikz}
\usepackage{cancel}
\usetikzlibrary{shapes,arrows, positioning, calc}
\tikzstyle{vertex}=[draw,fill=black!15,circle,minimum size=20pt,inner sep=0pt]
\tikzset{
  treenode/.style = {align=center, inner sep=0pt, text centered,font=\sffamily},
  arn_n/.style = {treenode, rectangle, white, font=\sffamily\bfseries, draw=black,fill=black, minimum width=2.5em,minimum height=1.5em},
  arn_r/.style = {treenode, circle, red, draw=red, minimum width=2.5em, very thick},
  arn_x/.style = {treenode, rectangle, draw=black,minimum width=0.5em, minimum height=0.5em}
}

\newcommand{\accapo}{\\\hspace*{1cm}\\}
\newcommand{\Eaccentata}{$\grave{\text{E}}$ }
\newcommand{\vopen}{``}
\newcommand{\vclose}{''}
\newcommand{\vclosespace}{'' }
\newcommand{\trasformata}{\xrightarrow{\mathscr{F}}}
\newcommand{\antitrasformata}{\xrightarrow{\mathscr{F}^{-1}}}
\newcommand{\intfity}{\int_{-\infty}^{+\infty}}

\setlength{\parindent}{0cm}
% chktex-file 44
% chktex-file 36
% chktex-file 1
\hfuzz=100pt

\title{Simulazione secondo esonero\\\normalsize fondamenti di telecomunicazioni}
\author{Flavio Colacicchi}
\date{19/01/2024}
\begin{document}
\maketitle
\begin{enumerate}
  \item \[x(t)=8sinc(4t)\cos(6\pi t)\]
    Campionato con intervallo \(T_c\) pari al valore massimo tale da non avere aliasing e viene poi ricostruito con un filtro passa-basso ideale nell'intervallo di frequenze \([-3,3]\) e ampiezza unitaria, calcolare
    \[\begin{matrix}
      T_c&&&x'(t)&&&X'(f)
    \end{matrix}\]
    Svolgimento:
    \[H(f)=rect\left(\frac{f}{6}\right)\]
    \[X(f)={\bcancel{8}}^{\bcancel{2}}\frac{1}{\bcancel{4}}rect\left(\frac{f}{4}\right)\frac{1}{\bcancel{2}}\left[\delta(f-3)+\delta(f+3)\right]=rect\left(\frac{f-3}{4}\right)\delta(f-3)+rect\left(\frac{f+3}{4}\right)\delta(f-3)\]
    \[X[n]=\frac{1}{T_c}\sum_{n}X\left(f-\frac{n}{T_c}\right)\]
    \[T_c=\frac{1}{f_c}=\frac{1}{2f_{\max}}=\frac{1}{10}\]
    Se lavorassi nel tempo:
    \[x_c(t)=x(t)\sum_{n}\delta(t-nT_c)=\sum_n x(n T_c)\delta(t-nT_c)\]
    Se lavorassi nel dominio della freuenza
    \[X_c(f)=10\sum_{n}X\left(f-10n\right)=10\sum_{n}rect\left(\frac{f-3-10n}{4}\right)+rect\left(\frac{f+3-10n}{4}\right)\]
    Il segbale campionato è quindi un treno di rect 
    \[X_c(f)=10\sum_n rect\left(\frac{f-5-10n}{8}\right)\]
    Ora lo faccio passare per il filtro
    \[X'(f)=X_c(f)H(f)=10\left[rect\left(\frac{f-2}{2}\right)+rect\left(\frac{f+2}{2}\right)\right]\]
    E ora antitrasformo
    \[x'(t)=10\cdot 2sinc(2t)(e^{i4\pi ft}+e^{-i4\pi ft})=40sinc(2t)\cos(4\pi t)\]
  \item \(x(t)\) viene campionato con il massimo intervallo \(T\) per non avere aliasing generando
    \[x_c(t)=\begin{cases}
      1&n=\pm 1\\
      0&\forall n\neq\pm 1
    \end{cases}\]
    Viene poi ricostruito con un filtro passa-basso ideale nell'intervallo \([-\frac{1}{2T},\frac{1}{2T}]\) e ha ampiezza \(T\). Calcolare
    \[\begin{matrix}
      x'(t)&&&X'(f)
    \end{matrix}\]
    Svolgimento:
    \[x_c(t)=\delta(t-T)+\delta(t+T)\trasformata X_c(f)=e^{-i2\pi fT}+e^{i2\pi fT}=2\cos(2\pi fT)\]
    \[H(f)=T\cdot rect\left(fT\right)\]
    \[X'(f)=2T\cos(2\pi fT)rect(fT)\]
    \[x'(t)=\bcancel{2}\bcancel{T}\frac{1}{\bcancel{2}}\frac{1}{\bcancel{T}}\left[\delta\left(t-T\right)+\delta\left(t+T\right)\right]*sinc\left(\frac{t}{T}\right)=sinc\left(\frac{t-T}{T}\right)+sinc\left(\frac{t+T}{T}\right)\]
    Si poteva anche usare l'enunciato del teorema del campionamento che ci dice che nel caso in cui la frequenza di campionamento sia maggiore o uguale a quella di Nyquist
    \[x'(t)=\sum_{n} x_c(t)sinc\left(\frac{t-nT}{T}\right)\]
    E dopo ciò trasformarlo per ottenere \(X'(f)\)
  \item \[z(t)=x(t)+y(t)\]
    Dove \(x\) e \(y\) sono due processi aleatori stazionari statisticamente indipendenti bianchi nella banda \([-W,W]\) con densità di probabilità uniforme in \([-A,A]\). Calcolare
    \[\begin{matrix}
      \mu_z&&P_Z&&R_Z(\tau)&&G_Z(z)&&p_Z(z)
    \end{matrix}\]
    Svolgimento:\accapo
    Essendo bianchi
    \[\begin{matrix}
      G_X(x)=\frac{P_X}{2W}rect\left(\frac{f}{2W}\right)&G_Y(y)=\frac{P_Y}{2W}rect\left(\frac{f}{2W}\right)
    \end{matrix}\]
    Essendo uniformi
    \[\begin{matrix}
      p_X(x)=\frac{1}{2A}rect\left(\frac{x}{2A}\right)&&p_Y(y)=\frac{1}{2A}rect\left(\frac{y}{2A}\right)
    \end{matrix}\]
    \[\mu_z=E[x(t)+y(t)]=E[x(t)]+E[y(T)]=\mu_x+\mu_y=\intfity x\cdot p_X(x)dx+\intfity y\cdot p_Y(y)dy=0+0=0\]
    \[P_Y=P_X=E[x^2(t)]=\intfity x^2 p_X(x)dx=\frac{1}{2A}\int_{-A}^A x^2dx=\frac{1}{6A}x^3\Big|_{-A}^A=\frac{A^2}{3}\]
    \[\Rightarrow G_X(x)=G_Y(y)=\frac{A^2}{6W}rect\left(\frac{f}{2W}\right)\]
    \[R_Z(\tau)=E[z(t+\tau)z(t)]=E[(x(t+\tau)+y(t+\tau))(x(t)+y(t))]=R_X(\tau)+R_Y(\tau)+2\mu_x \mu_y=R_X(\tau)+R_Y(\tau)\]
    \[G_Y(y)\antitrasformata R_Y(\tau)=R_X(\tau)=\frac{A^2}{\bcancel{6W}^3}\bcancel{2W}sinc(2W\tau)\Rightarrow R_Z(\tau)=\frac{2A^2}{3}sinc(2W\tau)\]
    \[R_Z(\tau)\trasformata G_Z(f)=\frac{2A^2}{3}\frac{1}{2W}rect\left(\frac{f}{2W}\right)\]
    \[p_Z(z)=p_X(x)*p_Y(y)\]
  \item \[y(t)=x(t-T)+x(t+T)\hspace{2cm} x(t)=\cos(2\pi f_0t+\varphi)\]
    sapendo \(f_0\) costante e \(\varphi\) uniformemente distribuita in \([0,2\pi]\)
    Calcolare 
    \[\begin{matrix}
      \mu_y=0&P_Y=2\cos^2(2\pi f_0T)\\
      R_Y(\tau)=2\cos(2\pi f_0\tau)\cos^2(2\pi f_0 T)&G_Y(f)=\cos^2(2\pi f_0 T)[\delta(f-f_0)+\delta(f+f_0)]
    \end{matrix}\]
\end{enumerate}
\end{document}