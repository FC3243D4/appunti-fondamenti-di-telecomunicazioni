\documentclass{article}
\usepackage{amsfonts}
\usepackage{amsmath}
\usepackage{amssymb}
\usepackage{mathrsfs}
\usepackage{extarrows}
\usepackage{hyperref}
\usepackage[utf8]{inputenc}
\usepackage{graphicx}
\usepackage{mathtools}
\usepackage[a4paper, total={6in, 8in}]{geometry}
\usepackage[table]{xcolor}
\usepackage{tikz}
\usepackage{cancel}
\usetikzlibrary{shapes,arrows}

\newcommand{\accapo}{\\\hspace*{1cm}\\}
\newcommand{\Eaccentata}{$\grave{\text{E}}$ }
\newcommand{\vopen}{``}
\newcommand{\vclose}{''}
\newcommand{\vclosespace}{'' }
\newcommand{\trasformata}{\xrightarrow{\mathscr{F}}}
\newcommand{\antitrasformata}{\xrightarrow{\mathscr{F}^{-1}}}

\setlength{\parindent}{0cm}
% chktex-file 44
% chktex-file 36
% chktex-file 1
\hfuzz=100pt

\title{Campionamento\\\normalsize fondamenti di telecomunicazioni}
\author{Flavio Colacicchi}
\date{29$-$/11/2023}
\begin{document}
\maketitle

Richiami delle serie di Fourier
\[\text{Treno di delta, treno campionatore, comb o pettine }\pi(t)=\sum_{n\in\mathbb{Z}}\delta(t-nT)=\sum_{k\in\mathbb{Z}}c_k e^{i2\pi n\frac{t}{T}}\]
Dove
\[c_k=\frac{1}{T}\int_{-\frac{T}{2}}^{\frac{T}{2}}\pi(t)e^{-i2\pi n\frac{t}{T}}dt=\frac{1}{T}\int_{-\frac{T}{2}}^\frac{T}{2}\delta(t)e^{-i2\pi n\frac{t}{T}}dt=\frac{1}{T}\int_{-\frac{T}{2}}^\frac{T}{2}\delta(t)dt=\frac{1}{T}\]
Ora voglio vedere il suo spettro
\[\Pi(f)=\begin{cases}
    \sum_{n\in\mathbb{Z}}e^{-i2\pi nTf}&\text{trasformando la definizione}\\
    \sum_{k\in\mathbb{Z}}\frac{1}{T}\delta\left(f-\frac{k}{T}\right)&\text{trasformando la serie di Fourier}
\end{cases}\]
\Large\[x(t)\to\boxed{D.A.C.}\to x_c(t)\]
\[x_c(t)=x(t)\pi(t)=\sum_{n\in\mathbb{Z}}x(t)\delta(t-nT)=\sum_{n\in\mathbb{Z}}x(nT)\delta(t-nT)\]\normalsize
Ora vedo cosa succende nel dominio delle frequenze
\[X(f)\to\boxed{D.A.C.}\to X_c(f)\]
\[X_c(f)=\begin{cases}
    X(f)*\Pi(f)=X(f)*\frac{1}{T}\sum_{n\in\mathbb{Z}}\delta\left(f-\frac{n}{T}\right)=\frac{1}{T}\sum_{n\in\mathbb{Z}}X\left(f-\frac{n}{T}\right)&\text{dalla serie di Fourier di \(\Pi(f)\)}\\
    \sum_{n\in\mathbb{Z}}x(nT)e^{-i2\pi nTf}=\sum_{n\in\mathbb{Z}}x[n]e^{-i2\pi nfT}&\text{operando prima sulle serie nel tempo}
\end{cases}\]
Quindi la trasformata di Fouerier di un segnale campionato o la trasformata della serie di un segnale campionato sono uguali
\[x[n]\trasformata X_c(f)=\sum_{n\in\mathbb{Z}}x[n]e^{-i2\pi fnT}\]
\[x[n]=x(nT)={x(t)}_{t=nT}\]
\[x(t)\trasformata X(f)\]
\[X_c(f)=\frac{1}{T}\sum_{n\in\mathbb{Z}}X\left(f-\frac{n}{T}\right)\]
Esempi:\begin{enumerate}
    \item \[x[n]=1\]
        \[X_c(f)=\begin{cases}
            \sum_{n\in\mathbb{Z}}e^{-i2\pi nfT}\\
            \frac{1}{T}\sum_{n\in\mathbb{Z}}X\left(f-\frac{n}{T}\right)\xlongequal{x(t)=1\trasformata X(f)=\delta(f)}\frac{1}{T}\sum_{n\in\mathbb{Z}}\delta\left(f-\frac{n}{T}\right)
        \end{cases}\]
    \item \[x[n]={\left[\frac{\sin(2\pi f_c nT)}{\pi n}\right]}^2=4 {f_c}^2 \pi^2 T^2 {sinc}^2(2f_c nT)\]
        \[x(t)=4{f_c}^2 T^2{sinc}^2(4f_c t)\trasformata 4{f_c}^2 T^2\frac{1}{2f_c}tri\left(\frac{f}{2f_c}\right)=X(f)\]
        \[X_c(f)=\sum_{n\in\mathbb{Z}}X\left(f-\frac{n}{T}\right)=2f_c T\sum_{n\in\mathbb{Z}} tri\left(\frac{f-nT}{2f_c}\right)\]
    \item \[y[n]=x[n]*h[n]=(a^n u[n])*(b^n u[n])\]
        Con la convoluzione
        \[y[n]=\begin{cases}
            \sum_{k\in\mathbb{Z}}x[k]h[n-k]=\sum_{k=0}^{n}a^{k}b^{n-k}=b^n\sum_{k=0}^{n}{\left(\frac{a}{b}\right)}^k=b^n\frac{1-{\left(\frac{a}{b}\right)}^{n+1}}{a-\frac{a}{b}}=\frac{b^{n+1}-a^{n+1}}{b-a}&n\geq0\\
            0&n<0
        \end{cases}\]
        \[y[n]=\frac{b^{n+1}-a^{n+1}}{b-a}u[n]\]
        Altrimenti posso fare
        \[Y_c(f)=X_c(f)H_c(f)\antitrasformata y[n]\]
        \[X_c(f)=\sum_{n=0}^{+\infty}a^n e^{-i2\pi nfT}=\sum_{n=0}^{+\infty}{(ae^{^{-i2\pi ft}})}^n=\frac{1}{1-ae^{-i2\pi fT}}\]
        \[H_c(f)=\frac{1}{1-be^{-i2\pi fT}}\]
        \[Y_c(f)=X_c(f)H_c(f)=\frac{1}{1-ae^{-i2\pi fT}}\cdot\frac{1}{1-be^{-i2\pi fT}}=\frac{A}{1-ae^{-i2\pi fT}}+\frac{B}{1-be^{-i2\pi fT}}=\]
        \[=\frac{A+B-(Ab+Ba)e^{-i2\pi fT}}{(1-ae^{-i2\pi fT})(1-be^{-i2\pi fT})}\Rightarrow\begin{cases}
            A+B=1\\Ab=-Ba
        \end{cases}\Rightarrow\begin{cases}
            B=\frac{b}{b-a}\\A=-\frac{a}{b-a}
        \end{cases}\Rightarrow\]
        \[\Rightarrow Y_c(f)=\frac{A}{1-ae^{-i2\pi fT}}+\frac{B}{1-be^{-i2\pi fT}}\antitrasformata y[n]=A a^n u[n]+B b^n u[n]=\frac{b}{b-a}b^n u[n]-\frac{a}{b-a} a^n u[n]=\]
        \[=\frac{b^{n+1}-a^{n+1}}{b-a}u[n]\]
    \item \[x[n]=-\frac{\sin\left(\frac{\pi n}{3}\right)}{\pi n}\cos\left(\frac{2\pi n}{6}\right)\hspace{2cm}T=1\]
        \[x(t)\frac{\sin\left(\frac{\pi t}{3}\right)}{\pi t}\cos\left(\frac{2\pi t}{6}\right)=\frac{1}{3}sinc\left(\frac{t}{3}\right)\cos\left(\frac{2\pi t}{6}\right)\]
        \[X(f)=rect(3f)*\left[\frac{1}{2}\delta\left(f-\frac{1}{6}\right)+\frac{1}{2}\delta\left(f+\frac{1}{6}\right)\right]=\frac{1}{2}rect\left[3\left(f-\frac{1}{6}\right)\right]+\frac{1}{2}rect\left[3\left(f+\frac{1}{6}\right)\right]\]
        \[X_c(f)=\frac{1}{T}\sum_{n\in\mathbb{Z}}X\left(f-\frac{n}{T}\right)=\sum_{n\in\mathbb{Z}}X(f-n)\]
    \item \[x[n]={\left[\frac{\sin\left(\frac{\pi n}{3}\right)}{\pi n}\right]}^2=\frac{1}{9}{sinc}^2\left(\frac{\pi n}{3}\right)\]
        \[X(f)=\frac{1}{3}tri(3f)\]
        \[X_c(f)=\sum_{n\in\mathbb{Z}}X(f-n)=\frac{1}{3}\sum_{n\in\mathbb{Z}}tri[3(f-n)]\]
\end{enumerate}

Se prendo \(T\) troppo grande rischio che i campioni in frequenza si vanno a sovrapporre creando così \textbf{aliasing} nella ricpstruzione

\end{document}