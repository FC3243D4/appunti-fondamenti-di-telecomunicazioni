\documentclass{article}
\usepackage{amsfonts}
\usepackage{amsmath}
\usepackage{amssymb}
\usepackage{amsmath}
\usepackage{mathrsfs}
\usepackage{extarrows}
\usepackage{hyperref}
\usepackage[utf8]{inputenc}
\usepackage{graphicx}
\usepackage{mathtools}
\usepackage[a4paper, total={6in, 8in}]{geometry}
\usepackage[table]{xcolor}
\usepackage{tikz}
\usetikzlibrary{shapes,arrows}

\newcommand{\accapo}{\\\hspace*{1cm}\\}
\newcommand{\Eaccentata}{$\grave{\text{E}}$ }
\newcommand{\vopen}{``}
\newcommand{\vclose}{''}
\newcommand{\vclosespace}{'' }

\setlength{\parindent}{0cm}
% chktex-file 44
% chktex-file 36
% chktex-file 1
\hfuzz=100pt

\title{esercizi prima parte del corso\\\normalsize fondamenti di telecomunicazioni}
\author{Flavio Colacicchi}
\date{24$-$25/10/2023}
\begin{document}
\maketitle 
\begin{center}classificazione di sistemi\end{center}
\begin{enumerate}
    \item \LARGE\[y(t)=|x(-t)|\]\normalsize
        \begin{itemize}
            \item lineare?
                \[x_1(t)\to y_1(t)=|x_1(-t)|\]
                \[x_2(t)\to y_1(t)=|x_2(-t)|\]
                \[ax_1(t)+bx_2(t)\to|ax_1(-t)+bx_2(-2)|\neq y_1(t)+y_2(t)\]
                \[\Downarrow\]
                \begin{center}no\end{center}
            \item tempo invariante?
                \[x(t-\tau)\to|x(-t-\tau)|\]
                \[y(t-\tau)=|x[-(t-\tau)]|=|x(t+\tau)|\neq |x(t-\tau)|\]
                \[\Downarrow\]
                \begin{center}no\end{center}
            \item causale?
                no perché \(y(t)\) non dipende dagli istanti precedenti a \(t\)
        \end{itemize}
        \item \LARGE\[y(t)=x^2(t-2)\]\normalsize
            \begin{itemize}
                \item lineare?
                    \[x_1(t)\to y_1(t)=x_1^2(t-2)\]
                    \[x_2(t)\to y_2(t)=x_2^2(t-2)\]
                    \[ax_1(t)+bx_2^t\to {(ax_1(t-2)+bx_2(t-2))}^2\neq y_1(t)+y_2(t)\]
                    \[\Downarrow\]
                    \begin{center}no\end{center}
                \item tempo invariante?
                    \[x(t-\tau)\to x^2(t-\tau-2)\]
                    \[y(t-\tau)\to x^2(t-2-\tau)\]
                    \[\Downarrow\]
                    \begin{center}sì\end{center}
                \item causale?\\
                    siccome l'uscita a ogni dato \(t\) dipende dall'istante \(t-2\) è causale
            \end{itemize}
            \item \LARGE\[y(t)=x(4-t)\]\normalsize
            \begin{itemize}
                \item lineare?
                    \[x_1(t)\to y_1(t)=x_1(4-t)\]
                    \[x_2(t)\to y_2(t)=x_2(4-t)\]
                    \[ax_1(t)+bx_2(t)\to ax_1(4-t)+bx_2(4-t)\]
                    \[\Downarrow\]
                    \begin{center}sì\end{center}
                \item tempo invariante?
                    \[x(t-\tau)\to x(4-t-\tau)\]
                    \[y(t-\tau)=x(4-t+\tau)\neq x(4-t-\tau)\]
                    \[\Downarrow\]
                    \begin{center}no\end{center}
                \item causale?\\
                no perché a ogni istante \(t\) il valore dell'uscita non dipende da valori pregressi bensì futuri
            \end{itemize}
\end{enumerate}
\begin{center}calcolo di energia e potenza\end{center}
\begin{enumerate}
    \item \LARGE\[x(t)=1+u(t)\]\normalsize
    non limitato nel tempo quindi probabilmente di potenza
        \[P_x=\lim_{\Delta t\to\infty}\frac{1}{\Delta t}\int_{-\frac{\Delta t}{2}}^{\frac{\Delta t}{2}}|x(t)|^2 dt=\lim_{\Delta t\to\infty}\frac{1}{\Delta t}\int_{-\frac{\Delta t}{2}}^{0}|x(t)|^2 dt+\lim_{\Delta t\to\infty}\frac{1}{\Delta t}\int_{0}^{\frac{\Delta t}{2}}|x(t)|^2 dt=\]
        \[\lim_{\Delta t\to\infty}\frac{1}{\Delta t}\int_{-\frac{\Delta t}{2}}^{0} dt+\frac{1}{\Delta t}\int_{0}^{\frac{\Delta t}{2}}4 dt=\lim_{\Delta t\to\infty}\frac{1}{\Delta t}\frac{\Delta t}{2}+\lim_{\Delta t\to\infty}\frac{1}{\Delta t}\frac{4\Delta t}{2}=\frac{5}{2}\]
    \item \LARGE\[x(t)=rect\left(\frac{t}{2T}\right)+tri\left(\frac{t}{T}\right)\]\normalsize
        \begin{center}grafico di tri traslata in alto di 1 nei punti in cui maggiore di 0\end{center}
        \[\Rightarrow x(t)=\begin{cases}2-\frac{|t|}{T}&-T\leq t\leq T\\0&\text{altrove}\end{cases}\]
        limitato nel tempo quindi probabilmente di energia
        \[E_x=\int_{-\infty}^{+\infty}|x(t)|^2dt=2\int_{0}^{T}{\left(2-\frac{t}{T}\right)}^2dt=2\int_{0}^{T}\left(4-\frac{4t}{T}+\frac{t^2}{T^2}\right)=2{\left[4t-\frac{2t^2}{T}+\frac{t^3}{3T^2}\right]}_0^T=\frac{14}{3}T\]
    \item \LARGE\[x(t)=rect\left(\frac{t}{T}\right)tri\left(\frac{t}{T}\right)\]\normalsize
        \begin{center}grafico di una tri da -T a T troncata a -T/2 e T/2\end{center}
        \[\Rightarrow x(t)=\begin{cases}1-\frac{|t|}{T}&-\frac{T}{2}\leq t\leq\frac{T}{2}\\0&\text{altrove}\end{cases}\]
        limitato nel tempo quindi probabilmente di energia
        \[E_x=2\int_{0}^{\frac{T}{2}}{\left(1-\frac{t}{T}\right)}^2dt=2\int_{0}^{\frac{T}{2}}\left(1-\frac{2t}{T}+\frac{t^2}{T^2}\right)dt=2{\left[t-\frac{t^2}{T}+\frac{t^3}{3T^2}\right]}_0^{\frac{T}{2}}=T-\frac{T}{2}+\frac{T}{12}=\frac{7}{12}T\]
    \item \LARGE\[\left[\cos\left(\frac{\pi t}{4}\right)+\sin\left(\frac{\pi t}{4}\right)\right]rect\left(t-\frac{1}{2}\right)\]\normalsize
        \begin{center}grafico della somma del seno e del coseno (una funzione sinusoidale passante per (-1,0),(0,1),(3,0) con apice valore massimo in 1 e minimo in 5 con periodo 8) troncata in 0 e 1\end{center}
        \[E_x=\int_{0}^{1}{\left[\cos\left(\frac{\pi t}{4}\right)+\sin\left(\frac{\pi t}{4}\right)\right]}^2dt=\int_{0}^{1}{\left(\cos^2\left(\frac{\pi t}{4}\right)+\sin^2\left(\frac{\pi t}{4}\right)+2\cos\left(\frac{\pi t}{4}\right)\sin\left(\frac{\pi t}{4}\right)\right)}^2dt=\]
        \[=\int_0^1 1+\sin\left(\frac{\pi t}{2}\right)dt=1-\frac{2}{\pi}{\left[\cos\left(\frac{\pi t}{2}\right)\right]}_0^1=1+\frac{2}{\pi}=\frac{\pi+2}{\pi}\]\newpage
    \item \LARGE\[\left[2\cos\left(\frac{\pi t}{\tau}\right)+i\sin\left(\frac{\pi t}{\tau}\right)\right]\]\normalsize
        \begin{center}periodo \(2\tau\)\end{center}
        segnale non limitato nel tempo quindi di potenza ma periodico quindi posso usare la formula
        \[P_x=\frac{1}{T}\int_{0}^{T}|x(t)|^2 dt=\frac{1}{2\tau}\int_{0}^{2\tau}4\cos^2\left(\frac{\pi t}{\tau}\right)+\sin^2\left(\frac{\pi t}{\tau}\right)dt=\frac{1}{2\tau}\int_{0}^{2\tau}1+3\cos^2\left(\frac{\pi t}{\tau}\right)dt=\]
        \[=1+\frac{3}{2\tau}\int_{0}^{2\tau}\cos^2\left(\frac{\pi t}{\tau}\right)dt=\frac{5}{2}\]
\end{enumerate}
\begin{center}convoluzione\end{center}
\begin{enumerate}
    \item \LARGE\[x(t)=\cos^2\left(\frac{2\pi t}{T}\right)*y(t)=rect\left(\frac{2 t}{T}\right)\]\normalsize
    \[x(t)*y(t)=\int_{-\infty}^{+\infty}x(\tau)y(t-\tau)d\tau=\int_{-\infty}^{+\infty}\cos^2\left(\frac{2\pi\tau}{T}\right)rect\left(\frac{2(t-\tau)}{T}\right)d\tau=\int_{t-\frac{T}{4}}^{t+\frac{T}{4}}\cos^2\left(\frac{2\pi\tau}{T}\right)d\tau=\]
    \[=\int_{t-\frac{T}{4}}^{t+\frac{T}{4}}\left[\frac{1}{2}+\frac{1}{2}\cos\left(\frac{4\pi\tau}{T}\right)\right]d\tau=\frac{1}{2}\frac{T}{2}=\frac{T}{4}\]
    \item \LARGE\[x(t)=\frac{1}{T}\left(t+\frac{T}{2}\right)rect\left(\frac{t}{T}\right)*y(t)=rect\left(\frac{t}{T}\right)\]\normalsize
    Con il metodo grafico vedo che $x(\tau)$ è una rampa che inizia in $-\frac{T}{2}$ dove vale 0 e vale 1 in $\frac{T}{2}$ e $y(t-\tau)$ è una rect che va da $t-\frac{T}{2}$ a $t+\frac{T}{2}$, quindi il loro prodotto vale
    \[\begin{cases}
        0&t<-T\\
        \frac{{(t+T)}^2}{2T}&-T<t\\
        \frac{T^2-t^2}{2T}&0<t<T\\
        0&t>T
    \end{cases}\]
    \[\frac{1}{T}\int_{-\frac{T}{2}}^{t+\frac{T}{2}}\tau+\frac{T}{2}d\tau=\frac{1}{T}\left(\frac{\tau^2}{2}+\frac{T}{2}\tau\right)\Big|_{-\frac{T}{2}}^{t+\frac{T}{2}}=\frac{1}{T}\left[\frac{{\left(t+\frac{T}{2}\right)}^2}{2}+\frac{T\left(t+\frac{T}{2}\right)}{2}-\frac{T^2}{8}+\frac{T^2}{4}\right]=\frac{{(t+T)}^2}{2T}\]
    \[\frac{1}{T}\int_{t-\frac{T}{2}}^{\frac{T}{2}}\tau+\frac{T}{2}d\tau=\frac{1}{T}\left(\frac{\tau^2}{2}+\frac{T}{2}\tau\right)\Big|_{t-\frac{T}{2}}^{\frac{T}{2}}=\frac{T^2-t^2}{2T}\]
\end{enumerate}
\end{document}