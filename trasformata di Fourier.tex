\documentclass{article}
\usepackage{amsfonts}
\usepackage{amsmath}
\usepackage{amssymb}
\usepackage{mathrsfs}
\usepackage{extarrows}
\usepackage{hyperref}
\usepackage[utf8]{inputenc}
\usepackage{graphicx}
\usepackage{mathtools}
\usepackage[a4paper, total={6in, 8in}]{geometry}
\usepackage[table]{xcolor}
\usepackage{tikz}
\usepackage{cancel}
\usetikzlibrary{shapes,arrows}

\newcommand{\accapo}{\\\hspace*{1cm}\\}
\newcommand{\Eaccentata}{$\grave{\text{E}}$ }
\newcommand{\vopen}{``}
\newcommand{\vclose}{''}
\newcommand{\vclosespace}{'' }
\newcommand{\trasformata}{\xrightarrow{\mathscr{F}}}
\newcommand{\antitrasformata}{\xrightarrow{\mathscr{F}^{-1}}}

\setlength{\parindent}{0cm}
% chktex-file 44
% chktex-file 36
% chktex-file 1
\hfuzz=100pt

\title{Trasformata Fourier\\\normalsize fondamenti di telecomunicazioni}
\author{Flavio Colacicchi}
\date{07$-$29/11/2023}
\begin{document}
\maketitle
\begin{center}
\LARGE\[x(t)\trasformata X(f)=\int_{-\infty}^{+\infty}x(t) e^{-i2\pi ft}dt\]
\[X(f)\antitrasformata x(t)=\int_{-\infty}^{+\infty}X(f) e^{i2\pi ft}df\]
\end{center}
Proprietà\footnote{tabella riassuntiva alla terzultima pagina del documento}:
\begin{enumerate}
    \item Duale
        \[X(t)\trasformata x(-f)\]
    \item Scala
        \[x(\alpha t)\trasformata\frac{1}{|a|}X\left(\frac{f}{a}\right)\]
    \item Traslazione nel tempo
        \[x(t-t_0)\trasformata e^{-2i\pi f t_0}X(f)\]
        Dimostrazione
        \[x(t-t_0)\trasformata \int_{-\infty}^{+\infty}x(t-t_0)e^{-2i\pi ft}dt\xlongequal{t'=t-t_0}\int_{-\infty}^{+\infty}x(t')e^{-2i\pi f(t'+t_0)}=e^{-2i\pi f t_0}\int_{-\infty}^{+\infty}x(t')e^{-2i\pi ft'}=\]
        \[=e^{-2i\pi f t_0}X(f)\]
    \item Traslazione in frequenza o della modulazione
        \[x(t)e^{2i\pi f_0 t}\trasformata \int_{-\infty}^{+\infty}x(t)e^{i2\pi f_0 t}e^{-2i\pi ft}=X(f-f_0)\]
    \item Derivazione
        \[x'(t)=\frac{d}{dt}x(t)\trasformata X'(f)=2i\pi fX(f)\]
        Dimostrazione
        \[x(t)=\int_{-\infty}^{+\infty}X(f)e^{2i\pi ft}\rightarrow\frac{d}{dx}x(t)=\int_{-\infty}^{+\infty}2i\pi fX(f)e^{2i\pi ft}df\]
    \item Duale rispetto alla derivata
        \[2i\pi t\cdot x(t)\trasformata\frac{dX(f)}{df}\]
        Dimostrazione\\
        Derivo la definizione di trasformata di Fourier
        \[\frac{d}{df}X(f)=\frac{d}{df}\int_{-\infty}^{+\infty}x(t)e^{-2i\pi t}df=\int-2i\pi f x(t)e^{^{-2i\pi ft}}\Rightarrow 2i\pi tx(t)\trasformata\frac{dX(f)}{df}\]
    \item Teorema della convoluzione
        \[x(t)*y(t)\trasformata X(f)Y(f)\]
        Dimostrazione
        \[x(t)*y(t)=\int_{-\infty}^{+\infty}x(t-\tau)y(\tau)d\tau\xLeftrightarrow{y(\tau)=\int_{-\infty}^{+\infty}Y(f)e^{2i\pi f\tau}df}\int_{-\infty}^{+\infty}x(t-\tau)\int_{-\infty}^{+\infty}Y(f)e^{2i\pi f\tau}dfd\tau\]
        Cambio l'ordine di integrazione
        \[\int_{-\infty}^{+\infty}x(t-\tau)\int_{-\infty}^{+\infty}Y(f)e^{2i\pi f\tau}d\tau df\xlongequal{t'=t-\tau}\int_{-\infty}^{+\infty}Y(f)e^{2i\pi ft}\int_{-\infty}^{+\infty}x(t')e^{-2i\pi ft'}dt'df=\]
        \[=\int_{-\infty}^{+\infty}Y(f)X(f)e^{2i\pi ft}df\]
    \item Integrazione
        \[\int_{-\infty}^{t} x(\tau)d\tau\trasformata \frac{X(0)}{2}\delta(f)+\frac{X(f)}{2i\pi f}\]
        Dimostrazione
        \[\int_{-\infty}^{t} x(\tau)d\tau=\int_{.\infty}^{+\infty}u(t-\tau)x(\tau)d\tau=u(t)*x(t)\trasformata\left(\frac{1}{2}\delta(f)+\frac{1}{2i\pi f}\right)X(f)\]
    \item Trasformata del complesso coniugato
        \[x^*(t)\trasformata X^*(-f)\]
        Dimostrazione
        \[x^*(t)\trasformata\int_{-\infty}^{+\infty}x^*(t)e^{-2i\pi ft}dt={\left(\int_{-\infty}^{+\infty}x(t)e^{2i\pi f}dt\right)}^*={(X(-f))}^*=X^*(-f)\leftarrow\text{simmetria Hermitiana}\]
    \item Correlazione
        \[x(t)\otimes y(t)\trasformata Y^*(f)X(f)\]
        Dimostrazione
        \[x(t)\otimes y(t)=y^*(-t)*x(t)\trasformata Y^*(f)X(f)\]
    \item Teorema di Parseval
        \[E_x=\int_{-\infty}^{+\infty}|X(f)|^2 df\]
        Dimostrazione
        \[z(t)x(t)*y(t)\trasformata Z(f)=X(f)Y(f)\Leftrightarrow z(t)=\int_{-\infty}^{+\infty}X(f)Y(f)e^{2i\pi ft}df\Rightarrow z(0)=\int_{-\infty}^{+\infty}X(f)Y(f)df\]
        facciamo lo stesso ragionamento ma con $y^*(-t)$
        \[z(t)=x(t)*y^*(-t)\trasformata Z(f)=X(f)Y^*(f)\Leftrightarrow z(t)=\int_{-\infty}^{+\infty}X(f)Y^*(f)e^{2i\pi ft}df\Rightarrow z(0)=\int_{-\infty}^{+\infty}X(f)Y^*(f)df=\]
        \[=<x(t),y(t)>\]
        scegliendo \(y(\tau)=x(\tau)\) vediamo che
        \[\int_{-\infty}^{+\infty}|x(\tau)|^2=\int_{-\infty}^{+\infty}|X(f)|^2 df\]
    \item Trasfomata di un segnale reale
        \[X(f)=\int_{-\infty}^{+\infty}x(t)e^{-2i\pi ft}dt\xlongequal{x\in\mathbb{R}}\int_{-\infty}^{+infty}x(t)\cos(2\pi ft)dt-i\int_{-\infty}^{+\infty}x(t)\sin(2\pi ft)dt=\Re (f)+i\Im (f)\]
        Allora
        \[\Re(f)=\Re(-f)\]
        \[\Im(f)=-\Im(f)\]
        E quindi
        \[X(-f)=\Re(-f)+i\Im(-f)=\Re(f)-i\Im(f)=X^*(f)\leftarrow\text{simmetria Hermitiana}\]
        Questo ci farà comodo se \(x(t)\in\mathbb{R}\) è pari, in questo caso
        \[\Im(f)=0\Rightarrow X(f)\in\mathbb{R}\text{ e pari}\]
        Mentre se è dispari
        \[\Re(f)=0\Rightarrow X(f)\in\mathbb{C}\setminus\mathbb{R}\text{ e dispari}\]
    \item Trasformata di un segnale periodico
        \[x(t)=\sum_{n\in\mathbb{Z}}c_n e^{2i\pi\frac{nt}{T}}\trasformata X(f)\xlongequal{\text{linearità}}\sum_{n\in\mathbb{Z}}c_n\delta\left(f-\frac{n}{T}\right)\]
    \item Legame con i coefficienti di Fourier\accapo
        i coefficienti di fourier possono essere calcolati come una trasformata di Fourier di un segnale uguale a quello originale ma diverso da 0 sono all'interno del suo periodo
        \[\dot{x}(t)=x(t)rect\left(\frac{t}{T}\right)\]
        \[c_n=\frac{1}{T}\int_{-\infty}^{+\infty}\dot{x}(t)e^{-2i\pi t\frac{n}{T}}dt=\frac{1}{T}\dot{X}\left(\frac{n}{T}\right)\]
\end{enumerate}

Trasformate notevoli\footnote{tabella riassuntiva all'ultima pagina del documento}
\begin{itemize}
    \item Delta
        \[\delta(t)\trasformata 1\]
    \item Delta traslata
        \[\delta(t-t_0)=e^{-2i\pi f t_0}\text{ per la proprietà di traslazione nel tempo}\]
    \item Rect
        \[rect\left(\frac{t}{T}\right)\trasformata Tsinc(Tf)\]
    \item Gaussiana
        \[x(t)=e^{-\alpha t}\trasformata \sqrt{\frac{\pi}{\alpha}}e^{-\frac{\pi^2 f^2}{\alpha}}\]
        Dimostrazione
        \[x(t)\trasformata X(f)=\int_{-\infty}^{+\infty}e^{-\alpha t}e^{-i2\pi ft}dt=\int_{-\infty}^{+\infty}e^{-\alpha\left[t^2+i\frac{2\pi ft}{\alpha}+{\left(\frac{i\pi f}{\alpha}\right)}^2-{\left(\frac{i\pi f}{\alpha}\right)}^2\right]}dt=e^{-\frac{\pi^2 f^2}{\alpha^2}\alpha}\int_{-\infty}^{+\infty}e^{-\alpha{\left(t+i\frac{\pi f}{\alpha}\right)}^2}dt\]
        \[\xRightarrow{t'=\sqrt{\alpha}\left(t+i\frac{\pi f}{\alpha}\right)}e^{-\frac{\pi^2 f^2}{\alpha^2}\alpha}\int_{-\infty}^{+\infty}e^{t'^2}\frac{dt'}{\sqrt{\alpha}}=\sqrt{\frac{\pi}{\alpha}}e^{-\frac{\pi^2 f^2}{\alpha}}\]
        Allora \textbf{la gaussiana su Fourier si autotrasforma}\accapo
        La gaussiana in forma canonica è
        \[e^{-\frac{t^2}{2\sigma^2}}\]
        Dove $\sigma$ è la deviazione standard, quindi abbiamo
        \[\alpha=\frac{1}{2\sigma^2}\]
    \item Esponenziale unilatero
        \[x(t)=e^{-\alpha t}u(t)\trasformata \frac{1}{\alpha+2\pi if}\]
        Dimostrazione
        \[\Rightarrow X(f)=\int_=^{+\infty}e^{-\alpha t}e^{-i2\pi ft}dt=\frac{e^{-(\alpha+2\pi if)t}}{-\alpha-2\pi if}\Bigg|_0^{+\infty}=\frac{1}{\alpha+2\pi if}\]
        Possiamo vedere che il modulo quadro è una Lorentziana
        \[|X(f)|^2=\frac{1}{\alpha^2+4\pi^2 f^2}\]
        \textbf{Esempi}:
        \begin{itemize}
            \item Possiamo sfruttare questa trasformata per fare quella dell'esponenziale bilatero
                \[x_1(t)=e^{-\alpha|t|}\]
                \[\Rightarrow x_1(t)=e^{-\alpha t}u(t)+e^{\alpha t}u(-t)=x(t)+x(-t)\trasformata X_1(f)=X(f)+X(-f)\text{ per la proprietà di scala}\]
                \[\Rightarrow X_1(f)=\frac{1}{\alpha+2i\pi f}+\frac{1}{\alpha-2i\pi f}=\frac{2\alpha}{\alpha^2 4\pi^2 f^2}\]
                Che è una Lorentziana canonica
            \item Invece quella del segnale
                \[x_2(t)=\frac{1}{\alpha+2i\pi t}\trasformata X_2(f)=e^{\alpha f}u(-f)\text{ per la proprietà di dualità}\]
            \item Gradino
                \[x_3(t)=u(t)=\lim_{\alpha\to0}e^{-\alpha t}u(t)\trasformata X_3(f)=\lim_{\alpha\to0}\frac{1}{\alpha+2i\pi f}=\frac{1}{2i\pi f}\]
                Che però non è del tutto corretto dato ce in 0 questo esplode a infinito quindi c'è una $\delta$, per trovarla lo rifaccio razionalizzando
                \[\frac{1}{\alpha+2i\pi f}=\frac{\alpha}{\alpha^2+e\pi^2 f^2}-\frac{2i\pi f}{\alpha^2+e\pi^2 f^2}\]
                Ne calcolo l'ampiezza
                \[A=\int_{-\infty}^{+\infty}\frac{\alpha}{\alpha^2+4\pi^2 f^2}df=\int_{-\infty}^{+\infty}\frac{1}{\alpha\left(+\frac{4\pi^2 f^2}{\alpha^2}\right)}\xlongequal{f'=\frac{2\pi f}{\alpha}}\frac{1}{\alpha}\int_{-\infty}^{+\infty}\frac{1}{1+f'^2}df'\frac{\alpha}{2\pi}=\frac{1}{2\pi}\arctan(f)\Big|_{-\infty}^{+\infty}=\]
                \[=\frac{1}{2\pi}\left[\frac{\pi}{2}-\left(-\frac{\pi}{2}\right)\right]=\frac{1}{2}\]
                \[\Rightarrow X_3(f)=\frac{1}{2}\delta(f)+\frac{1}{2i\pi f}\]
            \item Funzione segno
                \[x_4(t)=\begin{cases}
                    1&t>0\\
                    -1&t<0
                \end{cases}=u(t)-u(-t)\trasformata X_4(f)=\frac{1}{2}\delta(f)+\frac{1}{2i\pi f}-\frac{1}{2}\delta(-f)+\frac{1}{2i\pi f}\xlongequal{\delta(f)=\delta(-f)}\frac{1}{i\pi f}\]
        \end{itemize}
    \item Tri
        \[x(t)=tri\left(\frac{t}{T}\right)\trasformata\frac{\sin^2(\pi fT)}{\pi^2 f^2 T}\]
        Dimostrazione
        \[x(t)=tri\left(\frac{t}{T}\right)=\begin{cases}
            1-\left|\frac{t}{T}\right|&-T\leq t\leq T\\
            0&\text{altrove}
        \end{cases}\xRightarrow{\frac{d}{dx}}\begin{cases}
            \frac{1}{T}&0-T\leq t\leq 0\\
            -\frac{1}{T}&0\leq t\leq T\\
            0&\text{altrove}
        \end{cases}\leftarrow\text{due rect}\]
        \[\Rightarrow x'(t)=\frac{1}{T}rect\left(\frac{t+\frac{T}{2}}{T}\right)-\frac{1}{T}rect\left(\frac{t-\frac{T}{2}}{T}\right)\trasformata \frac{1}{\bcancel{T}}\bcancel{T}sinc(TF)e^{\bcancel{2}i\pi f\frac{T}{\bcancel{2}}}-\frac{1}{\bcancel{T}}\bcancel{T}sinc(TF)e^{-\bcancel{2}i\pi f\frac{T}{\bcancel{2}}}=\]
        \[=sinc(TF)\sin(\pi fT)2i=\frac{\sin^2(\pi fT)}{\pi fT}2i\xRightarrow{X(f)=2i\pi fX'(f)} X(f)=\frac{\sin^2(\pi fT)\bcancel{2i}}{\pi fT}\frac{1}{\bcancel{2i}\pi f}=\frac{\sin^2(\pi fT)}{\pi^2 f^2 T}\]
\end{itemize}
Prendiamo un filtro
\[x(t)\to\boxed{h(t)}\to y(t)=x(t)*h(t)\]
Ma è difficile usare un impulso per trobare la risposta inpulsiva quindi mandiamo all'inizio un segnale sinusoidale che va come un coseno
\[x(t)=e^{i2\pi f_0 t}\Rightarrow y(t)=\int_{-\infty}^{+\infty}h(\tau)e^{i2\pi f_0(t-\tau)}d\tau=e^{i2\pi f_0 t}\int_{-\infty}^{+\infty}h(\tau)e^{-i2\pi f_0 \tau}d\tau=H(f_0)e^{i2\pi f_0 t}\]
Dove \(H(f_0)\) è la \textbf{funzione di trasferimento} calcolata in \(f=f_0\), a questo punto possiamo usare il teorema della convoluzione per calcolare più facilmente l'uscita del filtro.\accapo
La trasformata di Fourier (o spettro del segnale) è spesso complessa, se ne studio il modulo sto studiando lo spettro in ampiezza mentre se ne studio la fase sto studiando lo spettro di fase del segnale.\accapo
Un segnale in banda base è un segnale intorno all'asse y mentre un segnale modulato è un segnale simmetrico rispetto a questo asse con un elemento intorno a $f_0$ e uno intorno a $-f_0$ dove $f_0$ è la \textbf{frequenza portante}.\accapo
Consideriamo un sistema tempo-discreto
\[x[n]\to\boxed{h[n]}\to y[n]\]
E prendiamo
\[x[n]=\delta[n]\to\boxed{h[n]}\to y[n]=h[n]\]
Quindi
\[y[n]=x[n]*h[n]\]
Supponiamo
\[x[n]=e^{2i\pi f_0 nT}\Rightarrow y[n]=h[n]*e^{2i\pi f_0 nT}=\sum_{k\in\mathbb{Z}}h[k]e^{2i\pi f_0(n-k)T}=e^{2i\pi f_0 nT}\sum_{k\in\mathbb{Z}}h[k]e^{-2i\pi f_0 kT}=\]
\[=e^{2i\pi f_0 nT}H(f_0)\]
Allora
\LARGE\[x[n]\trasformata X(f)=\sum_{n\in\mathbb{Z}}x[n]e^{-2i\pi fnT}\]\normalsize
Dove \(X(f)\) è periodica di periodo \(\frac{1}{T}\) infatti
\[X\left(f-\frac{1}{T}\right)=\sum_{n\in\mathbb{Z}}x[n]e^{-2i\pi \left(f-\frac{1}{T}\right)nT}=\sum_{n\in\mathbb{Z}}x[n]e^{-2i\pi fnT}e^{2i\pi n}\xlongequal{e^{2i\pi n}=1}\sum_{n\in\mathbb{Z}}x[n]e^{-2i\pi fnT}=X(f)\]
Spesso si usa una frequenza normalizzata per la rappresentazione
\[\phi=Tf\Rightarrow X(\phi)=\sum_{n\in\mathbb{Z}}e^{-2i\pi n\phi}\]
E questa è periodica di periodo 1, possiamo inoltre vedere che la definizione di trasformata di un segnale tempo-discreto ci da anche la serie di Fourier della trasformata i qui coefficienti sono il segnale di partenza, da qui possiamo vedere che
\LARGE\[X(f)\antitrasformata x[n]=T\int_{-\frac{1}{2T}}^{\frac{1}{2T}}X(f)e^{2i\pi nfT}df\]\normalsize\accapo
Proprietà\footnote{tabella riassuntva alla penultima pagina del documento}:
\begin{enumerate}
    \item Valor medio
        \[X(0)=\sum_{n\in\mathbb{Z}}x[n]\]
    \item duale del valor medio
        \[x[0]=T\int_{-\frac{1}{2T}}^{\frac{1}{2T}}X(f)df\]
    \item Traslazione nel tempo
        \[x[n-n_0]\trasformata X(f)e^{-2i\pi n_0 fT}\]
        \[x[n-n_0]\trasformata\sum_{n\in\mathbb{Z}}x[n-n_0]e^{-2i\pi fnT}\xlongequal{n'=n-n_0}\sum_{n'\in\mathbb{Z}}x[n']e^{-2i\pi fn'T}e^{-2i\pi n_0 fT}=X(f)e^{-2i\pi n_0 fT}\]
    \item Traslazione in frequenza
        \[x[n]e^{2i\pi f_0 nT}\trasformata X(f-f_0)\]
        \[x[n]e^{2i\pi f_0 nT}\trasformata\sum_{n\in\mathbb{Z}}x[n]e^{i2\pi f_0 nT}e^{-2i\pi fnT}=\sum_{n\in\mathbb{Z}}x[n]e^{-2i\pi(f-f_0)nT}=X(f-f_0)\]
    \item Teorema della convoluzione
        \[x[n]*h[n]\trasformata X(f)H(f)\]
        \[\sum_{k\in\mathbb{Z}}x[k]h[n-k]\trasformata \sum_{n\in\mathbb{Z}}\sum_{k\in\mathbb{Z}}x[k]h[n-k]e^{-i2\pi nfT}\xlongequal{n'=n-k}\sum_{k\in\mathbb{Z}}x[k]\sum_{n'\in\mathbb{Z}}h[n']e^{-2i\pi n'fT}e^{-2i\pi kfT}=\]
        \[=\sum_{k\in\mathbb{Z}}x[k]e^{-2i\pi kfT}H(f)=X(f)H(f)\]
    \item Opposta del teorema della convoluzione
        \[x[n]h[n]\trasformata T\int_{-\frac{1}{2T}}^\frac{1}{2T}X(\theta)H(f-\theta)d\theta\]
        \[x[n]h[n]\trasformata\sum_{n\in\mathbb{Z}}x[n]h[n]e^{-2i\pi nfT}=\sum_{n\in\mathbb{Z}}h[n]e^{-2i\pi nfT}T\int_{-\frac{1}{2T}}^{\frac{1}{2T}}X(\theta)e^{2i\pi n\theta T}d\theta=\]
        \[=T\int_{-\frac{1}{2T}}^\frac{1}{2T}X(\theta)\sum_{n\in\mathbb{Z}} h[n]e^{-2i\pi (f-\theta)nT}d\theta=T\int_{-\frac{1}{2T}}^\frac{1}{2T}X(\theta)H(f-\theta)d\theta\leftarrow\text{convoluzione circolare}\]

        Dimostriamo la convoluzioen circolare
        \[X(f)=\sum_{n\in\mathbb{Z}}\overline{X}\left(f-\frac{n}{T}\right)\hspace{2cm}H(f)=\sum_{k\in\mathbb{Z}}\overline{H}\left(f-\frac{k}{T}\right)\]
        \[\overline{X}(f)*\overline{H}(f)=\int_{-\infty}^{+\infty}\overline{X}(\theta)\overline(H)(f-\theta)d\theta\]
        \[X(f)\text{ convoluzione circolare }H(f)=T\int_{-\frac{1}{2T}}^{\frac{1}{2T}}X(\theta)H(f-\theta)d\theta=\]
        \[T\sum_{n\in\mathbb{Z}}\sum_{k\in\mathbb{Z}}\int_{-\frac{1}{2T}}^{\frac{1}{2T}}\overline{X}\left(\theta-\frac{n}{T}\right)\overline{H}\left(f-\theta-\frac{k}{T}\right)d\theta\xlongequal{\theta'=\theta-\frac{n}{T}}T\sum_{n\in\mathbb{Z}}\sum_{k\in\mathbb{Z}}\int_{-\frac{1}{2T}-\frac{n}{T}}^{\frac{1}{2T}-\frac{n}{T}}\overline{X}(\theta')\overline{H}\left(f-\theta'-\frac{k+n}{T}\right)d\theta'=\]
        \[=T\sum_{k\in\mathbb{Z}}\int_{-\frac{1}{2T}}^\frac{1}{2T}\overline{X}(\theta')\overline{H}\left(f-\theta'-\frac{k}{T}\right)d\theta'=T\sum_{k\in\mathbb{Z}}\int_{-\infty}^{+\infty}\overline{X}(\theta')\overline{H}\left(f-\theta'-\frac{k}{T}\right)d\theta'=T\sum_{k\in\mathbb{Z}}\overline{Y}\left(f-\frac{k}{T}\right)\]
    \item Trasformata delcomplessso coniugato
        \[y^*[n]\trasformata Y^*(-f)\]
        \[y^*[n]\trasformata=\sum_{n\in\mathbb{Z}}y^*[n]e^{-2i\pi nfT}={\left[\sum_{n\in\mathbb{Z}}y[n]e^{i2\pi nfT}\right]}^*=Y^*(-f)\leftarrow\text{simmetria Hermitiana}\]
    \item Teorema di Parseval
        \[E_x=T\int_{-\frac{1}{2T}}^\frac{1}{2T}|X(f)|^2df\]
        Dalla proprietà 6
        \[\sum_{n\in\mathbb{Z}}y[n]x[n]e^{-2i\pi nfT}=T\int_{-\frac{1}{2T}}^\frac{1}{2T}Y(\theta)X(f-\theta)d\theta\]
        Metto \(f=0\)
        \[\sum_{n\in\mathbb{Z}}y[n]x[n]=T\int_{-\frac{1}{2T}}^\frac{1}{2T}Y(\theta)X(-\theta)d\theta\Rightarrow\sum_{n\in\mathbb{Z}}y^*[n]x[n]=T\int_{-\frac{1}{2T}}^{\frac{1}{2T}}X(-\theta)Y^*(-\theta)d\theta\Rightarrow\sum_{n\in\mathbb{Z}}|x[n]|^2=\]
        \[=T\int_{-\frac{1}{2T}}^\frac{1}{2T}|X(f)|^2df=E_x\]
    \item Trasformata della correlazione
        \[R_{xy}[n]=x[n]\otimes y[n]=\sum_{k\in\mathbb{Z}}x[n-k]y^*[k]\trasformata X(f)Y^*(f)\]
        \[\sum_{k\in\mathbb{Z}}x[n-k]y^*[k]\trasformata\sum_{n\in\mathbb{Z}}\sum{k\in\mathbb{Z}}x[n-k]y^*[k]e^{-2i\pi nfT}\xlongequal{n'=n+k}\sum_{k\in\mathbb{Z}}y^*[k]\sum_{n'\in\mathbb{Z}}x[n']e^{-2i\pi n' fT}e^{2i\pi kfT}=\]
        \[=X(f)\sum_{k\in\mathbb{Z}}y^*[k]e^{2i\pi kfT}=X(f)Y^*(f)\leftarrow\text{crosscorrelazione}\]
    \item Trasformata dell'autocorrelazione
        \[x[n]\otimes x[n]=|X(f)|^2\]
\end{enumerate}
Esempi:
\begin{enumerate}
    \item \[x[n]=\delta[n]\]
        \[X(f)=\sum_{n\in\mathbb{Z}}\delta[n]e^{-2i\pi fnT}=1\]
    \item \[x[n]=\begin{cases}
        1&0\leq n<N\\
        0&\text{altrove}
    \end{cases}\leftarrow\text{rect tempo-discreta}\]
        \[X(f)=\sum_{n=0}^{N-1}1\cdot e^{-2i\pi fnT}=1+e^{-2i\pi ft}+e^{-4i\pi fT}+\cdots+e^{-i(N-1)2\pi fT}\xlongequal{1+x+x^2+\cdots+x^{N-1}=\frac{1-x^N}{1-x}}\frac{1-e^{-2i\pi NfT}}{1-e^{-2i\pi fT}}=\]
        \[=\frac{e^{-i\pi NfT}}{e^{-i\pi fT}}\frac{e^{i\pi fNT}-e^{-i\pi fNT}}{e^{i\pi fT}-e^{-i\pi fT}}=e^{-i\pi(N-1)fT}\frac{\sin(\pi fNT)}{\sin(\pi fT)}\]
        \[|X(f)|=\frac{\sin(\pi fNT)}{\sin(\pi fT)}\leftarrow\text{dinc}\leftarrow\text{digital sinc}\]
    \item \[x[n]=\delta[n]+\delta[n-1]\leftarrow\text{caso particolare dell'esempio precedente con }N=2\]
        \[X(f)=1+e^{-2i\pi fT}=e^{-i\pi fT}(e^{i\pi fT}+e^{-i\pi fT})=2e^{-i\pi fT}\cos(\pi fT)\leftarrow\text{filtro passa-basso}\]
        \[|X(f)|=2|\cos(\pi fT)|\]
    \item \[x[n]=\delta[n]-\delta[n-1]\]
        \[X(f)=1-e^{-2i\pi fT}=e^{-i\pi fT}(e^{i\pi fT}-e^{-i\pi fT})=2ie^{-i\pi fT}\sin(\pi fT)\leftarrow\text{filtro passa-alto}\]
    \item \[X(f)=rect\left(\frac{f}{2f_c}\right)\antitrasformata x[n]=?\]
        Ma \(X(f)\) non può essere la trasformata di una sequenza perché non è periodica, quindi dovrebbe essere un treno di rect, la riscrivo quindi così
        \[X(f)=\sum_{k\in\mathbb{Z}}rect\left(\frac{f+\frac{k}{T}}{2f_c}\right)\]
        E procedo
        \[x[n]=T\int_{-\frac{1}{2T}}^{\frac{1}{2T}}rect\left(\frac{f}{2f_c}\right)e^{i2\pi nfT}df=T\int_{-f_c}^{f_c}e^{2i\pi nfT}df=\bcancel{T}\frac{e^{2i\pi nfT}}{2i\pi n\bcancel{T}}\Bigg|_{-f_c}^{f_c}=\frac{\sin(2\pi nf_c T)}{\pi n}=2f_c Tsinc(2\pi f_c T)\]
\end{enumerate}

Richiami delle serie di Fourier
\[\text{Treno di delta o treno campionatore }\pi(t)=\sum_{n\in\mathbb{Z}}\delta(t-nT)=\sum_{k\in\mathbb{Z}}c_k e^{i2\pi n\frac{t}{T}}\]
Dove
\[c_k=\frac{1}{T}\int_{-\frac{T}{2}}^{\frac{T}{2}}\pi(t)e^{-i2\pi n\frac{t}{T}}dt=\frac{1}{T}\int_{-\frac{T}{2}}^\frac{T}{2}\delta(t)e^{-i2\pi n\frac{t}{T}}dt=\frac{1}{T}\int_{-\frac{T}{2}}^\frac{T}{2}\delta(t)dt=\frac{1}{T}\]
Ora voglio vedere il suo spettro
\[\Pi(f)=\begin{cases}
    \sum_{n\in\mathbb{Z}}e^{-i2\pi nTf}&\text{trasformando la definizione}\\
    \sum_{k\in\mathbb{Z}}\frac{1}{T}\delta\left(f-\frac{k}{T}\right)&\text{trasformando la serie di Fourier}
\end{cases}\]
\newpage\LARGE
\begin{center}
    Proprietà della trasformata di Fourier tempo-continua\\\scriptsize\hspace*{1cm}\\\large
    {\renewcommand{\arraystretch}{2}
        \begin{tabular}{|c|c|}
            \hline
            Proprietà&Enunciato\\
            \hline
            Duale&\(X(t)\trasformata x(-f)\)\\
            \hline
            Scala&\(x(\alpha t)\trasformata\frac{1}{|\alpha|}X\left(\frac{f}{\alpha}\right)\)\\
            \hline
            Traslazione nel tempo&\(x(t-t_0)\trasformata e^{-2i\pi ft_0}X(f)\)\\
            \hline
            \begin{tabular}{c}
                traslazione in frequenza\\
                o modulazione
            \end{tabular}&\(x(t)e^{2i\pi f_0 t}\trasformata X(f-f_0)\)\\
            \hline
            Derivazione&\(\frac{d}{dt}x(t)\trasformata2i\pi fX(f)\)\\
            \hline
            Duale della derivazione&\(2i\pi t\cdot x(t)\trasformata\frac{d}{df}X(f)\)\\
            \hline
            Teorema della convoluzione&\(x(t)*y(t)\trasformata X(f)Y(f)\)\\
            \hline
            Integrazione&\(\int_{-\infty}^t x(\tau)d\tau\trasformata \frac{X(0)}{2}\delta(f)+\frac{X(f)}{2i\pi f}\)\\
            \hline
            Complesso coniugato&\(x^*(t)\trasformata X^*(-f)\)\\
            \hline
            Correlazione&\(x(t)\otimes y(t)\trasformata X(f)Y^*(f)\)\\
            \hline
            Teorema di Parseval&\(E_x=\int_{-\infty}^{+\infty}|X(f)|^2 df\)\\
            \hline
            Trasformata di un segnale Reale&\(\begin{matrix}
                \Re(f)=\Re(-f)\\
                \Im(f)=-\Im(f)\\
                X(f)=X(-f)
            \end{matrix}\)\\
            \hline
            Trasformata di un segnale periodico&\(X(f)=\sum_{n\in\mathbb{Z}}c_n\delta\left(f-\frac{n}{T}\right)\)\\
            \hline
            Legame con i coefficienti di Fourier&\(\begin{matrix}
                \dot{x(t)}=x(t)rect\left(\frac{t}{T}\right)\\
                c_n=\frac{1}{T}\dot{X}\left(\frac{n}{T}\right)
            \end{matrix}\)\\
            \hline
        \end{tabular}
    }
\end{center}

\newpage\LARGE
\begin{center}
    Proprietà della trasformata di Fourier tempo-discreta\\\scriptsize\hspace*{1cm}\\\Large
    {\renewcommand{\arraystretch}{2}
        \begin{tabular}{|c|c|}
            \hline
            Proprietà&Enunciato\\
            \hline
            Valor medio&\(X(0)=\sum_{n\in\mathbb{Z}}x[n]\)\\
            \hline
            Duale del valoer medio&\(x[0]=T\int_{-\frac{1}{2T}}^{\frac{1}{2T}}X(f)df\)\\
            \hline
            Traslazione nel tempo&\(x[n-n_0]\trasformata X(f)e^{-2i\pi n_0 fT}\)\\
            \hline
            Traslazione in frequenza&\(x[n]e^{2i\pi f_0 nT}\trasformata X(f-f_0)\)\\
            \hline
            Teorema della convoluzione&\(x[n]*y[n]\trasformata X(f)Y(f)\)\\
            \hline
            Opposta del toerema della convoluzione&\(x[n]y[n]\trasformata T\int_{-\frac{1}{2T}}^\frac{1}{2T}X(\theta)Y(f-\theta)d\theta\)\\
            \hline
            Trasformata del complesso coniugato&\(x^*[n]\trasformata X^*(-f)\)\\
            \hline
            Teorema di Parseval&\(E_x=T\int_{-\frac{1}{2T}}^\frac{1}{2T}|X(f)|^2df\)\\
            \hline
            Trasformata della correlazione&\(x[n]\otimes y[n]\trasformata X(f)Y^*(f)\)\\
            \hline
            Trasformata dell'autocorrelazione&\(x[n]\otimes x[n]=|X(f)|^2\)\\
            \hline
        \end{tabular}
    }
\end{center}

\newpage\huge
\begin{center}
    Trasformate notevoli\\\scriptsize\hspace*{1cm}\\\huge
    {\renewcommand{\arraystretch}{2}
        \begin{tabular}{|c|c|}
            \hline
            \(x(t)\)&\(X(f)\)\\
            \hline
            \(\delta(t)\)&1\\
            \hline
            \(\delta(t-t_0)\)&\(e^{-2i\pi f t_0}\)\\
            \hline
            \(rect\left(\frac{t}{T}\right)\)&\(Tsinc(Tf)\)\\
            \hline
            \(e^{-\alpha t}\)&\(\sqrt{\frac{\pi}{\alpha}}e^{-\frac{\pi^2 f^2}{\alpha}}\)\\
            \hline
            \(e^{-\alpha t}u(t)\)&\(\frac{1}{\alpha+2i\pi f}\)\\
            \hline
           \(e^{-\alpha|t|}\)&\(\frac{2\alpha}{\alpha^2 4\pi^2 f^2}\)\\
            \hline
            \(\frac{1}{\alpha+2i\pi f}\)&\(e^{\alpha f}u(-f)\)\\
           \hline
            \(u(t)\)&\(\frac{1}{2}\delta(f)+\frac{1}{2i\pi f}\)\\
            \hline
            \(u(t)-u(-t)\)&\(\frac{1}{i\pi f}\)\\
            \hline
            \(tri\left(\frac{t}{T}\right)\)&\(\frac{\sin^2(\pi fT)}{\pi^2 f^2 T}\)\\
            \hline
        \end{tabular}
    }
\end{center}

\end{document}